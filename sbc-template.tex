\documentclass[12pt]{article}

\usepackage{sbc-template}

\usepackage{graphicx,url}

%\usepackage[brazil]{babel}   
\usepackage[utf8]{inputenc}  
     
\sloppy

\title{Banco de dados em tempo real: Firebase}

\author{Fábio da Silva Takaki\inst{1}, Lucas Martins Valladares Ribeiro\inst{1} }


\address{Faculdade de Ciências e Tecnologia\\
  Universidade Estadual Paulista \\
  "Júlio de Mesquita Filho" \\
  Caixa Postal 19060-900  -- Presidente Prudente -- SP -- Brasil
  \email{fabio@takaki.me, lucasmbtos@live.com}
}

\begin{document} 

\maketitle

\begin{abstract}
   With the emergence of the digital society and the Internet of Things, several concepts for data storage and manipulation have emerged with the purpose of reducing resource and maintenance costs of which Cloud Computing stands out. This article proposes a case study of a tool that addresses one of the service models of the Cloud Computing (IaaS, PaaS and SaaS) architecture, which is Firebase, a tool created by Google. Firebase is a mobile platform that provides a number of services for the rapid development of high-quality mobile applications, from which stand out the real-time database service that will be the focus of this work.
\end{abstract}
     
\begin{resumo} 
   Com o surgimento da sociedade digital e a Internet das Coisas, diversos conceitos para armazenamento e manipulação dos dados surgiram com a finalidade de diminiuir os custos de recursos e manutenção do qual destaca-se o Cloud Computing. Este artigo propõe um estudo de caso de uma ferramenta que aborda um dos modelos de serviço da arquitetura Cloud Computing (IaaS, PaaS e SaaS) que é o Firebase, uma ferramenta criada pela Google. O Firebase é uma plataforma mobile que fornece diversos serviços para o rápido desenvolvimento de aplicativos móveis de alta qualidade, do qual destaca-se o serviço de banco de dados em tempo real que será o foco principal deste trabalho.
\end{resumo}


\section{Introdução}

%  descrição do cenário e problemas em aberto
Com a popularização da internet em que surgiu uma sociedade digital, o volume de informações se propagou de forma abundante nos últimos anos, principalmente com o avanço das tecnologias móveis. Em decorrência disso, diversos conceitos para armazenamento e manipulação dos dados surgiram com a finalidade de diminuir os custos de recursos e manutenção. Nesse sentido surgiram diversos serviços para suprir a escalabilidade de recursos como \cite{WA} \cite{GAE} \cite{AWS} entre outros serviços que seguem conceitos de Cloud Computing que serão abordados mais a frente neste artigo. No entanto, até então não haviam serviços que são voltados para o desenvolvimento móvel ou web no qual oferece um armazenamento e sincronização de dados em tempo real, que são recursos que são essenciais para aplicativos como chat ou redes sociais atualmente. Assim, este artigo apresenta um estudo de caso do Firebase, que tem por finalidade fornecer um serviço de dados em nuvem em tempo real, utilizado para criação de aplicativos multiplataformas que compartilham um banco de dados não relacional que a própria ferramenta disponibiliza. Como consequência, possibilita os desenvolvedores criarem aplicativos de alta qualidade de forma rápida e fácil, sem a necessidade de se preocupar com infraestrutura e escalabilidade de recursos.
\\
Este artigo está organizado da seguinte maneira: na Seção II será apresentado uma fundamentação teórica; na Seção III será apresentado a metodologia de desenvolvimento aqui aplicada; na Seção IV um estudo de caso da ferramenta firebase mostrando suas vantagens e desvantagens; por fim na Seção V resultados e trabalhos futuros.

\section{Fundamentação Teórica} \label{sec:firstpage}

\subsection{Cloud Computing}

Cloud computing é uma tecnologia em que os recursos de hardware e software, tais como aplicações especiais, CPU, armazenamento e muitos outros são fornecidos aos usuários através da Internet como um serviço, e é cobrado com base no que você usa \cite{1}. Esses recursos têm a capacidade de escalabilidade automática de acordo com a demanda do cliente.

O conceito de Cloud Computing é classificado em três modelos de serviço: Software as a Service (SaaS), Plataform as a Service (PaaS) e Infrastructure as a Service (IaaS). Com isso, é possível criar uma representação dos modelos de serviço por meio de camadas de virtualização (Figura~\ref{fig:architeture}).

\begin{figure}[ht]
\centering
\includegraphics[width=.9\textwidth]{architeture.png}
\caption{Arquitetura Cloud Computing (retirado de: \cite{2})}
\label{fig:architeture}
\end{figure}

\textbf{Infrastructure as a Service}

Infraestrutura como serviço, tradução literal de Infrasctructure as a Service, compõe a base das camadas de virtualização em que é responsável por oferecer hardware como serviço, como o próprio nome diz.

\textbf{Product as a Service}
É proporcionar um ambiente ou plataforma apropriada na qual o
desenvolvedor pode criar as aplicações e o software através da Internet sem necessidade de instalação ou gerenciar o ambiente de desenvolvimento \cite{1}. Seguindo a figura~\ref{fig:architeture}, um exemplo para este modelo de serviço seria a Google App Engine, na qual oferece uma plataforma de desenvolvimento para aplicações web e móvel.

\textbf{Software as a Service}

Esse modelo propõe um software como serviço fornecido através da internet, não necessitando de sua instalação. O fornecedor deste modelo de serviço, é responsável por controlar e limitar o uso das aplicações. Portanto, o cliente deste modelo de serviço também fica livre da necessidade de um infraestrutura do qual o fornecedor já disponibiliza.

Assim, com os conceitos de Cloud Computing, o Firebase é considerado Product as a Service, por ser uma plataforma de desenvolvimento que oferece armazenamento e sincronização de dados em tempo real, utilizando apenas código client-side.

\section{Metodologia de Apoio ao Firebase}

O Firebase Realtime Database é um banco de dados hospedado em nuvem. Os dados são armazenados como JSON e sincronizados em tempo real para cada cliente conectado. Ao criar aplicativos multiplataforma com nossos SDKs para iOS, Android e JavaScript, todos os seus clientes compartilham uma instância do Realtime Database e recebem automaticamente atualizações com os dados mais recentes. \cite{Firebase}

\section{Estudo de Caso}

Section titles must be in boldface, 13pt, flush left. There should be an extra
12 pt of space before each title. Section numbering is optional. The first
paragraph of each section should not be indented, while the first lines of
subsequent paragraphs should be indented by 1.27 cm.

\section{Conclusões e Trabalhos Futuros}

The subsection titles must be in boldface, 12pt, flush left.

\section{Figures and Captions}\label{sec:figs}


Figure and table captions should be centered if less than one line
(Figure~\ref{fig:architeture}), otherwise justified and indented by 0.8cm on
both margins, as shown in Figure~\ref{fig:exampleFig2}. The caption font must
be Helvetica, 10 point, boldface, with 6 points of space before and after each
caption.

\begin{figure}[ht]
\centering
\includegraphics[width=.3\textwidth]{fig2.jpg}
\caption{This figure is an example of a figure caption taking more than one
  line and justified considering margins mentioned in Section~\ref{sec:figs}.}
\label{fig:exampleFig2}
\end{figure}

In tables, try to avoid the use of colored or shaded backgrounds, and avoid
thick, doubled, or unnecessary framing lines. When reporting empirical data,
do not use more decimal digits than warranted by their precision and
reproducibility. Table caption must be placed before the table (see Table 1)
and the font used must also be Helvetica, 10 point, boldface, with 6 points of
space before and after each caption.

\begin{table}[ht]
\centering
\caption{Variables to be considered on the evaluation of interaction
  techniques}
\label{tab:exTable1}
\includegraphics[width=.7\textwidth]{table.jpg}
\end{table}

\section{Images}

All images and illustrations should be in black-and-white, or gray tones,
excepting for the papers that will be electronically available (on CD-ROMs,
internet, etc.). The image resolution on paper should be about 600 dpi for
black-and-white images, and 150-300 dpi for grayscale images.  Do not include
images with excessive resolution, as they may take hours to print, without any
visible difference in the result. 

\section{References}

Bibliographic references must be unambiguous and uniform.  We recommend giving
the author names references in brackets, e.g. Blablabla said Nobody.

The references must be listed using 12 point font size, with 6 points of space
before each reference. The first line of each reference should not be
indented, while the subsequent should be indented by 0.5 cm.

\bibliographystyle{sbc}
\bibliography{bibliografia}

\end{document}
